\chapter{Strømforsyning}
Ud fra kravspecifikation skal delmodulerne (SM og VBTE) kunne drives fra en 24V AC/DC forsynings kilde. Selve modulerne er designet til 12V DC og 5V DC forsyningsspændringer , der designes derfor en strømforsyning der regulere spændingen så den kan levere 12V 1A og 5V 0,5A.  

\section{Overordnet design}
I dette afsnit beskrives og vises det overordnede hardware blokdiagram over strømforsyningen samt beskrivelse signaler.

\begin{figure}[H]
\centering
\includegraphics[width=1\textwidth]{billeder/PowerSupplyBlok}
\caption{Overordnet blokdiagram for strømforsyning}
\label{fig:PowerSubbly Blok}
\end{figure}
\newpage
\subsection{Blokke}
Nedenfor beskrives de enkelte blokke illustreret på \textit{Figur~\ref{fig:PowerSubbly Blok}}
\subsubsection{Sikring}
Sikringen beskytter forsyningskilde, hvis der bliver for stor belastning på strømforsyningen
\subsubsection{Dobbelt ensretter}
Ensretter AC eller DC spændingen fra forsyningskilden.
\subsubsection{Udglatning}
Udglatter det ensrettet signal til en stabil positiv DC. 
\subsubsection{Regulator 12V}
Regulere den ensrettet DC ned til 12V DC.
\subsubsection{Regulator 5V}
Regulere den ensrettet DC ned til 5V DC.
\newpage
\section{Nedbrydning af blokke}
For at gøre designet af de forskellige blokke mere overskuelig nedbrydes de enkle blokke.
\subsection{Sikring}
