\section{Design og Implementering}
\label{ch:DesignImplementering}
%%%%%%%%%%%%%%%%%%%%%%%%%%%%
%%%%%%%%%% VBTE %%%%%%%%%%%%
%%%%%%%%%%%%%%%%%%%%%%%%%%%%
\subsection{VBTE}
I dette afsnit beskrives design og implementering af VBTE for både software og hardware. 
\subsubsection{Hardware}
Til VBTE'en er der blevet designet hardware til at styre de to ventiler samt den keramiske ultralydstransmitter og receiver. Hardware er blevet udfærdiget i to dele. Den ene er programeret hardware på PSoC'en, den anden er hardware uden for PSoC'en. Hardware designprocessen til VBTE'en gik igennem 3 faser:
\begin{enumerate}
\item Overordnet design
\item Nedbrydning af blokke
\item Opbygning af design
\end{enumerate}
Gennem disse faser er designet blevet udfærdiget. Fremgangsmåden er anvendt for at overskueligtgøre systemet og lette arbejdet ved at dele systemet op i små dele. På \textit{figur \ref{fig:HWVBTE}} ses det overordnede design af VBTE'en. Der vil i rapporten tages udgangspunkt i ventilkredsen samt receiverdriveren på PSoC'en.
\begin{figure}[H]
\centering
\includegraphics[width = 0.9\textwidth]{billeder/HWVBTE}
\caption{Illustrering af overordnet design af hardware på VBTE.}
\label{fig:HWVBTE}
\end{figure}
\subsubsection{Hardware på PSoC'en}
Hardware opbygget på PSoC'en kunne uden problemer være blevet designet uden for PSoC'en, men PSoC miljøet gør det meget nemt at arbejde med mange forskellige elementer. På figur \ref{fig:HWVBTE} ses hardware designet på PSoC'en.
\begin{figure}[H]
\centering
\includegraphics[width=0.7\textwidth]{billeder/PSOC_Hardware}
\caption{Hardware på PSoC'en.}
\label{fig:PSOCHW}
\end{figure}
\subsubsection{Software}


%%%%%%%%%%%%%%%%%%%%%%%%%%%%
%%%%%%%%%%% SM %%%%%%%%%%%%%
%%%%%%%%%%%%%%%%%%%%%%%%%%%%
\subsection{SM}
I dette afsnit beskrives design og implementering af SM modulet. SM modulet består af både software og hardware.
\subsubsection{Hardware}
SM modulets hardware består af en konverteringskreds og en PSoC. På PSoC'en er monteret et Kionix KXSC7-2050 accelerometer. Konverteringskredsen anvendes til at sende UART signaler fra PSoC til en KI modulet. Accelerometerets x-akse anvendes til hældningsmålinger for hældningssensorblokken. Designfasen til SM er delt op i 3 faser:
\begin{enumerate}
\item Overordnet design
\item Nedbrydning af blokke
\item Opbygning af design
\end{enumerate}
Denne fremgangsmåde gør det muligt for en udefrakommende at følge med i processen og at kunne implementere modulet så det overholder de krav der er stillet. Ligeledes gør fremgangsmåden det lettere at overskue flere løsninger til hvert problem. på \textit{Figur~\ref{fig:SMHW1}} ses det Overordnede design og på \textit{Figur~\ref{fig:SMPSOC1}} ses PSoC blokken i SM. Der bliver efterfølgende taget udgangspunkt i Hældningssensoren. \\
\begin{figure}[H]
	\centering
	\begin{minipage}[b]{0.48\textwidth}\centering
	\includegraphics[width=0.80\textwidth]{billeder/SMHardware}
	\end{minipage}
	\begin{minipage}[b]{0.48\textwidth}\centering
	\includegraphics[width=0.80\textwidth]{billeder/SMPSoCblock}
	\end{minipage}
	\begin{minipage}[t]{0.48\textwidth}
	\caption{Hardware blok for SM}
	\label{fig:SMHW1}
	\end{minipage}
	\begin{minipage}[t]{0.48\textwidth}
	\caption{PSoC blok for sm}
	\label{fig:SMPSOC1}
	\end{minipage}
\end{figure}
Hældningssensoren består af 2 komponenter: det førnævnte accelerometer samt en DelSig ADC internt i PSoC'en. Valget af accelerometer kommer af at have lavet en række prototyper der ikke mødte vores krav, hvilket accelerometeret i PSoC'en gjorde. Valget af ADC faldt på en DelSig da, den er meget støj immun grundet det indbyggede lavpas filter og har en høj opløsning. Valgte komponenter er illustret på \textit{Figur~\ref{fig:levelsensor}}. ADC konverterer det analoge signal fra, en pin forbundet til, accelerometer til en digital værdi der så senere bliver anvendt i softwaren. For ADC og accelerometerets opsætning se da afsnit \ref{ch:DetajlDesign}~\textit{Detaljeret hardware design} i Bilag.
\begin{figure}[htbp]
	\centering
	\includegraphics[width=0.50\textwidth]{billeder/levelsensor}
	\caption{Hældningssensorens implementering}
	\label{fig:levelsensor}
\end{figure}
\subsubsection{Software}
SM software behandler den konverterede data fra ADC'en samt kommunikation med VBTE og KI modulernerne. Sofwarens udvikling har fulgt de samme faser som hardwaren, hvilket har ført til letforståelig og læselig kode. Softwaren er illustreret på \textit{Figur~\ref{fig:SMKD}}. Der vil efterfølgende blive taget udgangspunkt i funktionerne autoReg og getFromKI.
\begin{figure}[H]
	\centering
	\includegraphics[width=0.75\textwidth]{billeder/smKlassediagram}
	\caption{Klassediagram for SM}
	\label{fig:SMKD}
\end{figure}
\textbf{getFromKI}\\
Funktionen er bedst beskrevet med et flowdiagram:
