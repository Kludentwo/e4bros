\chapter{Afgrænsning}
\label{ch:afgraensning}
\textit{En oversigt over de afgrænsninger dette projekt er udarbejdet med.}
\subsubsection{Afgrænsning givet:}
Gruppen skal fremstille et system der overholder nogle krav. Kravene er at systemet skal kunne interagere med omverdenen vha. af sensorer og aktuatorer. Endvidere skal der også anvendes faglige elementer fra fjerde semesterets kurser. Transmission af data mellem enheder i projektet skal være pålidelig. Til slut skal systemet indeholde brugerinteraktion. 
\subsubsection{Afgrænsninger sat af gruppen selv}
Gruppen udarbejdede i starten af projektet ideer til en række funktionaliteter. Disse funktionaliteter blev inddelt i henholdsvis grundsystem og udvidelser, som vist i nedenfor.
\begin{itemize}
\item \textbf{Grundsystem:}
\begin{itemize}
\item[$\diamond$] Elektronisk måling af hældning
\item[$\diamond$] Automatisk regulering af niveau i ballasttanke
\item[$\diamond$] Niveaumåling i ballasttanke
\item[$\diamond$] Advarselssignaler
\end{itemize}
\item \textbf{Udvidelser:}
\begin{itemize}
\item[$\diamond$] Måling af afstand til terminal-kaj
\item[$\diamond$] Måling af dybdegang
\item[$\diamond$] Manuel styring af ballastniveau
\item[$\diamond$] Pålidelig kommunikation med ekstern enhed
\end{itemize}
\end{itemize}
Grundsystemet er det system gruppen har fastlagt sig på at implementere i projektforløbet. Funktionaliteterne her er valgt for at opfylde basiskrav fra opgaveformuleringen se \textit{\ref{ch:Opgaveformulering}}.
Udvidelser er nedprioriterede funktionaliteter da de enten har lille relevans eller ikke er kritiske for at systemets grundformål kan opfyldes egen projektformulering, se \textit{\ref{ch:Projektformulering}}. De vil derfor kun blive prioriteret såfremt tiden er dertil. 
Man vil senere hen ligeledes kunne udvide systemet med funktionaliteter på baggrund af feedback fra kunden og markedsundersøgelser.

