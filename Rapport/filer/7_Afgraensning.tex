\chapter{Afgrænsning}
\textit{En oversigt over de afgrænsninger dette projekt er udarbejdet med.}\\
Afgrænsninger sat på forhånd:\\
\begin{enumerate}[a]
\item Systemet skal interagere med omverdenen vha. sensorer og aktuatorer.
\item Systemet skal omfatte pålidelig datatransmission.
\item Systemet skal have brugerinteraktion.
\end{enumerate}
Afgrænsninger sat af gruppen selv:\\
Som en del af dette projekt har vi opdelt komponenter i to grupper: Grundsystem og udvidelser. Grundsystemet er det basale system med de funktionalitet der skaber et færdigt produkt. Udvidelser er funktionaliter der kunne være rare at have eller ville kunne øge værdien af produktet. Disse kan dog også være funktionalitet som kunden ikke har gavn af og derfor er ubrugelige. Grundsystem og udvidelser ses i \textit{Tabel \ref{tabel:Grundogudvid} }
\begin{table}[H]
\centering
\begin{tabular}{|l|l|}
\hline
\textbf{Grundsystem:} &Elektronisk måling af hældning\\
 &Automatisk regulering af niveau i ballasttanke\\
 &Niveaumåling i ballasttanke\\
 &Mulighed for brugerinteraktion\\
 &Advarselssignaler\\\hline
\textbf{Udvidelser:} &Måling af afstand til terminal-kaj\\
 &Måling af dybdegang\\
 &Manuel styring af ballast niveau\\
 &Pålidelig kommunikation med ekstern enhed\\\hline
\end{tabular}
\caption{Grundsystem og Udvidelser til BROS}
\label{tabel:Grundogudvid}
\end{table}