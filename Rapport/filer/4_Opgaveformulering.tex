\chapter{Projektformulering}
\section{Opgaveformulering}
\label{ch:Opgaveformulering}
Opgaveformuleringen er givet til gruppen for projektet som retningslinier. Følgende er givet som krav i disse retningslinier.
\begin{itemize}
\item Systemet skal interagere med omverdenen vha. sensorer og aktuatorer.
\item Der skal anvendes relevante faglige elementer fra semestrets kurser.
\item Systemet skal omfatte pålidelig transmission af data mellem udvalgte enheder.
\item Systemet skal kunne interagere med en bruger
\end{itemize}
\section{Projektformulering}
\label{ch:Projektformulering}
\subsubsection{Baggrund}
Når man laster eller losser et bulkskib bruges der mange resourcer på at kontrollere at skibet ikke får slagside. Der bruges lang tid på at planlægge hvor, hvornår og hvad der skal lastes. Hertil står skibsføreren på broen under hele lastningen/losningen og overvåger at det bliver gjort korrekt. Inde på terminalen bliver skibet også overvåget for at sikre, at der ikke sker fejl. Derudover er det ifølge lovgivningen\footnote{Se bilaget \textit{Jura} for relevant lovgivning og henvisning til lovskrifter} også muligt at imødekomme en slagsside, forsaget af en forestående last, ved at flytte rundt på ballast i ballasttanke. Gruppen vil derfor gerne arbejde med et system der automatisk kan afhjælpe problemer med slagsside og minimere tiden, der skal bruges på planlægning af en lastning/losning.

\subsubsection{Projektdefinition}
Det er gruppens ønske at lave et system der korrigerer skibe fra at få slagside i forbindelse med lastning og losning af gods. Overordnede systemkrav udarbejdet i projektformuleringen:
\begin{itemize}
\item Systemet skal logge data omkring skibsnavn og status til en database.
\item Systemet skal monitorere om skibet er i vatter.
\item Systemet skal modkompensere hvis skibet måles til at være ude af vatter.
\end{itemize}
