\chapter{Referencer}
\section{Artefakter}
\subsection{Kravspecifikation}
Kravspecifikationsdokumentet er udarbejdet i begyndelsen af projektet og omfatter beskrivelse af Use Cases, ikke funktionelle krav samt kvalitetsfaktorer. Den fuldstændige kravspecifikation kan se i bilag. (Kravspecifikation.pdf)

\subsection{Accepttestspecifikation}
Accepttestspecifikationsdokumentet beskriver de tests der skal laves for at undersøge om de ønskede krav er opfyldt. Den fuldstændige accepttestspecifikation kan ses i bilag (Accepttest.pdf).

\subsection{Systemarkitektur}
Systemarkitektur dokumentet beskriver systemets HW/SW opbygning og grænseflader. Den fuldstændige systemarkitektur kan ses i bilag (Systemarkitektur.pdf).

\subsection{Integrationstestspecifikation}
Integrationtestspecifikation beskriver de test der skal laves for at undersøge hvorledes de forskellige komponenter kan kommunikere. Den fuldstændige Integrationstest kan ses i bilag (Integrationstest.pdf).

\subsection{Detaljeret design}
\label{ch:DetajlDesign}
Det detaljerede design dokument beskriver hvordan HW/SW er designet og hvordan systemets komponenter fungerer. Det fuldstændige Detaljeret design dokument kan ses i bilag (Detaljeret Hardware design.pdf og Detaljeret Software design.pdf).

\subsection{Enhedstestspecifikation}
Enhedstestspecifikation beskriver de tests der skal laves for at undersøge om de forskellige stubbe af systemet fungere hensigtsmæssigt. Den fuldstændige enhedstestspecifikation kan ses i bilag (Enhedstest.pdf).

\section{Hjemmesider}
http://www.docs.google.com\\
http://office.microsoft.com/en-us/visio/\\
http://www.maplesoft.com/\\
http://www.ni.com/multisim/\\
http://kurser.iha.dk/eit/eit-lab/lagerliste.htm\\
https://www.retsinformation.dk/Forms/R0710.aspx?id=26449\\
http://www.apache.org/\\
http://www.mysql.com/\\

\section{Liste over bilag på CD}
Komponentliste.pdf\\
SCRUM.xls\\
Logbog.pdf\\
\subsection{Kode}
\subsubsection{KI}
hest
\subsubsection{SM}
hestning
\subsubsection{VBTE}
honning
\subsubsection{Databasen}
\subsection{Dokumentation}
Accepttest.pdf\\
Arkitektur.pdf\\
Detaljeret\_hardware\_design.pdf\\
Detaljeret\_software\_design.pdf\\
Enhedstest.pdf\\
Integrationstest.pdf\\
Kravspecifikation.pdf\\

\subsection{Datablade}
\begin{table}[H]
\begin{tabular}{|l|l|}
\hline
\textbf{Datablad:}& \textbf{Type:} \\\hline
CY8C55 & PSoC5 \\\hline
KXSC7-2050& Accelerometer\\\hline
ST3232cn & Level konverter \\\hline
BD139 & Transistor \\\hline
BC547 & Transistor \\\hline
400SRT 160 & Ultralydstransmitter og receiver \\\hline
Ventil & Ventil til ventilspolen \\\hline
Ventilspole & Ventilspole der driver ventilen \\\hline
LM317 & Spændingsregulator \\\hline
kbl04 & Diodebro \\\hline

\end{tabular}
\end{table}


\subsection{Billeder}
hvis vi har billeder af vores produkt!\\

\subsection{Jura}
Teknisk forskrift om lastning og losning af bulkskibe.\\
Bilag til foreskriften
\begin{table}[H]
\begin{tabular}{|p{1.5cm}|p{12cm}|} \hline
\cellcolor[gray]{0.85}Nummer: & \cellcolor[gray]{0.85}Beskrivelse:  \\ \hline
7 & Bulkskibes og deres besætnings sikkerhed kan forbedres ved at mindske risikoen for, at de lastes eller losses uhensigtsmæssig ved terminaler for fast bulklast. Det kan gøres ved at etablere harmoniserede procedurer for samarbejde og kommunikation mellem skib og terminal og ved at stille egnethedskrav til skibe og terminaler.   \\ \hline
9 & Bulkskibe, der anløber terminaler med henblik på lastning eller losning af fast bulklast, bør være egnede til dette formål. Terminalerne bør ligeledes være egnede til at modtage og laste eller losse anløbne bulkskibe. Til disse formål er der fastsat egnethedskriterier i BLU-koden. \\ \hline
10 & Terminalerne bør med det formål at forbedre samarbejde og kommunikation med skibsførerne om lastning og losning af fast bulklast udpege en terminalrepræsentant, der har ansvaret for lastning og losning i terminalen, og stille informationshæfter med terminalens og havnens krav til rådighed for skibsførerne, jf. bestemmelserne i BLU-koden. \\ \hline
12 &  For at sikre, at lastning og losning nøje bliver forberedt, aftalt og udført på en sådan måde, at skibets eller besætningens sikkerhed ikke kan bringes i fare, bør skibsførerens og terminalrepræsentantens ansvar fastlægges. Til dette formål findes der bestemmelser i den internationale konvention af 1974 om sikkerhed for menneskeliv på søen (SOLAS-konventionen af 1974), IMO-resolution A.862(20) og BLU-koden. Til samme formål kan der med udgangspunkt i disse internationale instrumenter ligeledes fastsættes procedurer for, hvordan lastning og losning forberedes, aftales og udføres.\\ \hline
\end{tabular}
\end{table}

Artikel 2 Anvendelsesområde\\
Direktivet finder anvendelse på:
\begin{table}[H]
\begin{tabular}{|p{1.5cm}|p{12cm}|} \hline
\cellcolor[gray]{0.85}Nummer: & \cellcolor[gray]{0.85}Beskrivelse:  \\ \hline
1 & alle bulkskibe, uanset flag, som anløber en terminal med henblik på lastning eller losning af fast bulklast, og   \\ \hline
2 & alle terminaler i medlemsstaterne, som anløbes af bulkskibe, der er omfattet af dette direktiv. \\ \hline
\end{tabular}
\end{table}

BILAG IV skibsførerens pligter før og under lastning og losning (som omhandlet i artikel 7, nr. 1, litra d)) \\
Før og under lastning og losning skal skibets fører sørge for følgende:\\
\begin{table}[H]
\begin{tabular}{|p{1.5cm}|p{12cm}|} \hline
\cellcolor[gray]{0.85}Nummer: & \cellcolor[gray]{0.85}Beskrivelse:  \\ \hline
1 & Skibets ansvarshavende officer skal lede lastning og losning af ladningen og udtømning og indtag af ballastvand. \\ \hline
2 & Fordelingen af ladning og ballastvand skal overvåges under hele laste- eller losseprocessen for at sikre, at skibets konstruktion ikke overbelastes. \\ \hline
3 & Skibet skal holdes på ret køl; kræves der af driftsmæssige årsager en vis slagside, skal den holdes så lille som mulig. \\ \hline
7 & Terminalrepræsentanten skal gøres opmærksom på behovet for afpasning af deballastning eller ballastning og laste- og losserater for skibet og på enhver afvigelse fra deballastnings- eller ballastningsplanen og andre forhold, der kan have betydning for lastning eller losning af ladningen. \\ \hline
\end{tabular}
\end{table}