\chapter{Forord}
\label{ch:forord}
Denne rapport er udarbejdet af fem ingeniørstuderende ved Aarhus Universitet, Ingeniørhøjskolen. Rapporten er hovedproduktet i et obligatorisk projektforløb på 4. semester og gennemgår overordnet gruppens besvarelse og gennemførelse af projektet. Ud over rapporten er der blevet udarbejdet en række projektdokumentationsdokumenter og et fysisk produkt. For yderligere detaljer henvises der til projektdokumentationen.

Det har været gruppens formål med projektet at lære udviklingsværktøjerne bedre at kende. Systemet er derfor anset som det emne, redskaberne anvendes på, mere end det egentlig mål med projektet.
 
Rapporten er skrevet med henblik på at læseren er af samme faglige niveau som gruppen, hvilket vil afspejle sig i det faglige sprog. 

\chapter{Indledning}
Projektets emne er "slagsideregulering af bulkskib", som er et selvvalgt emne. Rapporten beskriver udviklingsprocessen herunder projektforløbet, hvilke metoder der er anvendt og hvilke overvejelser der ligger til grund for de valgte løsninger. I forbindelse med, at de valgte løsninger bliver beskrevet vil der også blive fremlagt alternativer og begrundelser for at de ikke blev valgt. 

Der har fra ekstern side været stillet nogle krav til projektet og det system der skulle udvikles. Arbejdsprocessen og nogle af komponenterne er således påvirket af disse krav. Kravene kan ses i \ref{{ch:Opgaveformulering}}.

Formålet med projektet er at anvende de teorier og metoder, som er blevet tilegnet gennem studiet, og ikke mindst tilegne sig ny viden på egen hånd, for at fuldføre gennemførelsen af et komplet projektforløb.

Projektet har været inddelt i fem udviklingsfaser. Udviklingsfaserne er som følger:
\begin{itemize}
\item Kravspecifikation
\item Analyse og arkitektur
\item Detaljeret design
\item Implementering
\item Test
\end{itemize}


\section{Læsevejledning}
Rapportens opbygning er struktureret således at den giver den bedste gennemgang af hele projektforløbet. Rapporten er i hovedtræk delt op i to dele. De første afsnit beskriver det overordnede system og projekt. Tilblivelsen af systemet og projektet med tilhørende overvejelser beskrives i de efterfølgende afsnit. Denne adskillelse sker mellem afsnit \textit{\ref{ch:afgraensning}} og \textit{\ref{ch:projektbeskrivelse}}. Alle afsnit er skrevet så de som udgangspunkt godt kan stå alene, hvorfor der igennem rapporten vil komme gentagelser hvis man læser denne fortløbende. Rapportens egentlige indhold begynder fra afsnit \textit{\ref{ch:Opgaveformulering}} – opgaveformulering.  Her gennemgås opgaveformuleringen. Opgaveformuleringen indeholder minimumskrav til projektet og er givet til gruppen af vejlederen. I projektformuleringen bliver der defineret præcist hvad dette projekt kommer til at dreje sig om, og hvordan gruppen har formuleret dette. Herefter følger en beskrivelse af det samlede, tænkte system.
I afsnit \textit{\ref{ch:kravspecikikation}} – Kravspecifikation, fremlægges kravene der fra gruppen er stillet til projektet. Herefter beskrives projektafgrænsningen samt arbejdsmetoder og fremgangsmåde i afsnit \ref{ch:afgraensning} til \textit{\ref{ch:metoder}}.\\
Afsnit \textit{\ref{ch:analyse}} beskriver det analyse arbejde projektet har gennemgået. I afsnittene fra \textit{\ref{ch:systemarkitektur}} og \textit{\ref{ch:DesignImplementering}} nedbrydes hele projektet fra øverste abstraktionsniveau og ned til implementering. Der er her gået i dybden med de vigtige aspekter i forhold til dette projekt. Disse afsnit har samtidig også en naturlig overgang til hinanden ud fra systemarkitekturen. \\
Hernæst samles der op på de opnåede resultater i afsnit \textit{\ref{ch:Resultater}}. Efter resultaterne er præsenteret, fremlægges de erfaringer gruppen har opnået igennem hele projektforløbet, samt hvilke ting der har fungeret godt. Afslutningsvist konkluderes der på hele projektet på godt og ondt i afsnit \textit{\ref{ch:konklusion}} for til sidst at sætte nogle ord på forbedringer til systemet i afsnit \textit{\ref{ch:Forbedringer}}. Det anbefales dog at læse rapporten fortløbende for at få den bedste samlede forståelse for projektet og produktet.