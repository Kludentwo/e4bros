\chapter{Forord}
Denne rapport er udarbejdet af fem ingeniørstuderende ved Aarhus Universitet, Ingeniørhøjskolen. Rapporten er hovedproduktet i et obligatorisk projektforløb på 4. semester og gennemgår overordnet gruppens besvarelse og gennemførelse af projektet. Derudover er der blevet lavet en række projektdokumentationsdokumenter og et fysisk produkt. For yderligere detaljer henvises der til projektdokumentationen.
 
Rapporten er skrevet med henblik på at læseren er af samme faglige niveau som gruppen, hvilket vil afspejle sig i det faglige sprog. 

\chapter{Indledning}
Denne rapport omhandler 4. semesters projekt. Projektets emne er "slagsideregulering af bulkskib", som er et selvvalgt emne. Rapporten beskriver processen af projektet herunder hvordan forløbet har været, hvilke metoder der er anvendt og hvilke overvejelser der ligger til grund for de valgte løsninger. I forbindelse med, at de valgte løsninger bliver beskrevet vil der også blive fremlagt alternativer og begrundelser for at de ikke blev valgt.

Der har været stillet enkelte krav til projektet og det system der skulle udvikles. Nogle af disse krav afspejler derfor også hvordan arbejdsprocessen har været og ikke mindst nogle af de komponenter som indgår i det færdige system. 

Formålet med projektet er at anvende de teorier og metoder, som er blevet tilegnet gennem studiet, og ikke mindst tilegne sig ny viden på egen hånd, for at fuldføre gennemførelsen af et komplet projektforløb.

Projektet har været inddelt i et antal udviklingsfaser. Udviklingsfaserne er som følger:
\begin{itemize}
\item Kravspecifikation
\item Analyse og arkitektur
\item Detaljeret design
\item Implementering
\item Test
\end{itemize}


\section{Læsevejledning}
Rapportens opbygning er struktureret således at den giver den bedste gennemgang af hele projektforløbet. Rapporten er i hovedtræk delt op i 2 dele. De første afsnit beskriver det overordnede system og projekt. Tilblivelsen af systemet og projektet med tilhørende overvejelser beskrives i de efterfølgende afsnit. Denne adskillelse sker mellem afsnit 9 og 10\fxnote{lav dynamiske referencer}. Alle afsnit er skrevet så de som udgangspunkt godt kan stå alene, hvorfor der igennem rapporten vil komme gentagelser hvis man læser denne fortløbende. Rapportens egentlige indhold begynder fra afsnit 6 – opgaveformulering.  Her gennemgås opgaveformuleringen givet fra vejledere til gruppen, som indeholder minimumskrav til projektet. I projektformuleringen bliver der defineret præcist hvad dette projekt kommer til at dreje sig om, og hvordan gruppen har formuleret dette. Herefter følger en beskrivelse af det samlede, tænkte system.
I afsnit 8 – krav, fremlægges kravene der fra gruppen er stillet til projektet. Herefter beskrives projektafgrænsningen samt arbejdsmetoder og fremgangsmåde i afsnit 8 og 10.1.
Afsnit 10.2 beskriver det analyse arbejde projektet har gennemgået. I afsnittene fra 10.2-10.5 nedbrydes hele projektet fra øverste abstraktion og ned til implementering. Der er her gået i dybden med de vigtige aspekter i forhold til dette projekt. Disse afsnit har samtidig også en naturlig overgang til hinanden ud fra systemarkitekturen. 
Hernæst samles der op på de opnåede resultater i afsnit 10.6. Efter resultaterne er præsenteret, fremlægges de erfaringer gruppen har opnået igennem hele projektforløbet, samt hvilke ting der har fungeret godt. I afsnit 10.8 snakkes der ganske kort om de anvendte udviklingsværktøjer. Afslutningsvist konkluderes der på hele projektet på godt og ondt i afsnit 11. Det anbefales dog at læse rapporten fortløbende for at få den bedste samlede forståelse for projektet og produktet.
\fxnote{Lave refrencerne til de enkelte afsnit.}