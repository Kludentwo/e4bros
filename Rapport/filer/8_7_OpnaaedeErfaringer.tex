\section{Opnåede erfaringer}
\label{ch:OpXP}
%%%%%%%%%%%%%%
%\subsection{bulletpoints til OpXP}
%\item Nyt samarbejde og nye gruppemedlemmer (Hvor godt vi har sammensat en ny gruppe)
%\item Hvor mange iterationer vi har været igennem (Forbedringer af start design). Igen vigtigheden af at man altid kan gå tilbage og at dokumentation ikke er støbt i beton.
%\item Opnåede erfaring omkring faserne (Hvor vigtige de er og hvordan vi efterfølgende har haft det nemmere på faserne)
%\item Samarbejde med kunden. Inddrage kunden(Carl) i processen.
%\item Forståelse for test (Test er ikke til for bare at vise systemet virke men for at afsløre væsentlige mangler, hvilket har forbedret projektet (og produktet)).
%\item Uddeligering af opgaver (Noget johnny sagde med at det var vigtigt der var en ansvarsperson bag hver modul).
%\item Udvikling (Sammenkogt erfaring af Johnny og Nicolai)
%\end{itemize}
%%%%%%%%%%%%%%%

\subsection{Gruppen}
Det har været rigtig interessant at skulle samarbejde fem personer fra fire forskellige sammentømrede grupper. Fordi gruppen er ny har vi skulle finde en fælles måde at gøre tingene på. De gamle metoder er blevet sammenholdt og forskellene har givet anledning til spørgsmål; spørgsmål der ikke tidligere er blevet stillet. Diskussionerne det har medført har været mange, lange, og til tider frustrerende, men ens for dem er at de har været utrolig lærerige. De diskussioner har medført at gruppens forståelse af udviklingsprocessen samt hvad der er vigtigt når man strukturer et produkt, har rykket sig væsentlig mere end det har gjort sig gældende på de foregående semestre - og det uanset hvilken gruppe man tidligere har været i.\\
Gruppen er blevet udfordret på kommunikationen internt. Alle har skullet vænne sig til hvordan de nye gruppemedlemmer arbejder, hvad deres forcer er og ikke mindst hvad deres svagheder er.\\
I udviklingsforløbet er systemets moduler blevet delt op i ansvarsområder. Gruppen har erfaret at dette er en force, da det sikrer et tilhørelsesforhold og dermed også et ansvar for hver enkelt moduls opgaver blev udført.

\subsection{Agile udviklingsmetoder}
Gruppen har gennem dette projekt forløb fået en bedre forståelse for iterationer og hvordan dokumentation holdes opdateret efter behov. Endvidere har gruppen erfaret vigtigheden i omstilling og muligheden for arbejde iterativt.\\
Gruppen har arbejdet med faserne i projekt og har opnået en bedre forståelse for vigtigheden af faser og iterationer. Især kravspecifikation og systemarkitektur er igennem iterationer blevet forbedret, hvilket efterfølgende har gjort implementering og tests lettere at håndtere og gennemføre.\\


\subsection{Udvikling af nye Komponenter}
Gennem udvikling af projektet har gruppen erfaret, at udviklingen af nye moduler kan være så tidskrævende at det ofte kan svare sig at købe færdiglavede komponenter og inkorporere dem i systemet for at sikre en overordnet funktion. Udviklingstiden for et modul vurderes ud fra teknologiundersøgelsen og da gruppen ikke har prøvet at lave et komponent helt fra bunden er dette en ny erfaring. Derudover kan der spares betydelig tid og købte produkter kan biddrage til et pålideligt system da disse komponenter er langt mere gennemtestede end hvad vi overhovedet kan nå.

\subsection{Test}
Fra tidligere projekter har gruppen tænkt tests som noget der bare skulle laves. Dette har medført et ambivalent forhold til test. I dette projekt har gruppen erfaret at test er med til at fremhæve mulige forbedringer af den implementering, der er udført. Testresultater medtages i den iterative designprocedure, med henblik på at forbedre det testede modul. Denne erfaring har ført til mere tilfredstillende prototyper igennem projektforløbet.

\subsection{værktøjer}
SVN har været anvendt igennem projektet med stor succes. Det har gjort arbejdet nemmere når der skulle skrives dokumentation og rapport. Derudover har gruppen eksperimentet med at anvende latex til at skrive rapport og doukmentation.



%\textbf{Agilere udviklingsværktøjer - en hjælp i læringsprocessen}\\
%I løbet af projektet har gruppen i ryk fået en bedre forståelse af udviklingsmetoderne - hver gang med ændringer i dokumentationen til følge. De mange gennemgange har lært gruppen at jo længere man er i processen, jo flere steder skal der rettes til, når der kommer ændringer i systemet. Dog hjælper iterationen også med forståelsen af systemet. Gruppen har ikke haft nogen forventning om at dokumentationen er korrekt første gang den er skrevet - det kan slet ikke lade sig gøre. Tværtimod har gruppen lært at iterationer hjælper til at finde fejl og mangler i dokumentationen, fordi man tvinges til at gennemlæse den, når der skal rettes i den.\\
%Gruppen har været mange iterationer igennem, og hver iteration har været nemmere at implementere end den foregående som følge af det styrkede kendskab til systemet. Samtidig bliver man for hver iteration mødt med faserne endnu engang. Jo flere møder med faserne, jo bedre forståelse. Således er iterationerne også en hjælp til forståelse af udviklingsprocessen.
%
%
%\textbf{At udvikle et system med kunden i centrum}\\
%Som en del af vejledermøderne har vejleder ageret som en potentiel kunde. Den tilgang til projektet har været ny for gruppens medlemmer og har givet en ny indfaldsvinkel til kravspecifikationen. Det at kravspecifikationen er aftalen med kunden giver en grænse for hvad kravspecifikationen skal indeholde og hvad den ikke skal indeholde. Det er en grænse, gruppens medlemmer har manglet på de tidligere semestre.
%
%\section{Agile udviklingsmetoder - din ven i nøden aka. chip og chaps homeboy}
%Gruppen har gennem projektforløbet lært hvordan iterationer er med til at hæve kvaliteten af dokumentationen og forbedre systemet. Der er opnået erfaringer omkring vigtigheden i at kunne omstille sig og ændre sin måde at gøre tingene på. Alt dette for at forbedre slutproduktet og dokumentationen.\\
%Den højnede forståelse har medvirket til at gruppen har kunne sammenfatte test af højere kvalitet.
%Gruppen har været mange iterationer igennem, og hver iteration har været nemmere at implementere end den foregående som følge af det styrkede kendskab til systemet. Samtidig bliver man for hver iteration mødt med faserne endnu engang. Jo flere møder med faserne, jo bedre forståelse. Således er iterationerne også en hjælp til forståelse af udviklingsprocessen.
