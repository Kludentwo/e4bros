\section{Projektgennemførelse}
Projektet er udført af en gruppe på fem personer. Gruppen er ny og sammensat af personer fra fire forskellige projektgrupper. Grundet gode erfaringer fra de forskellige grupper har gruppen valgt at arbejde med projektet ud fra Scrum-modellen. Gruppen har fastlagt rollerne i tabel \ref{table:roller}. \\
Projektet er blevet brudt ned i mindre dele og det enkelte gruppemedlem har fået ansvaret for hver sin del. Der har så internt været sparring mellem de dele, der har kommunikation med hinanden. Nedenfor ses fordelingen af arbejdet:
\begin{table}[H]
\centering
\begin{tabular}{ll}
Kontrolinterfacet:			&Rasmus Lund-Jensen\\
Webserver og database: 		&Mick Holmark\\
Strømforsyning:				&Jacob Roesen\\
Vandballasttankenhed:		&Johnny Kristensen\\
Styringsmodulet:			&Nicolai Glud\\
\end{tabular}
\caption{Tabel over ansvarsfordeling}
\label{table:ansvar}
\end{table}
Der er blevet udarbejdet en tidsplan, der løbende er blevet revirderet. Projektets overordnede tidsplan for faserne og de eksterne milestones ligger som bilag.\\
\subsection{Rollefordelinger}
\begin{table}[H]
\centering
\begin{tabular}{ll}
Projektleder: 			&Jacob Roesen\\
Projektkoordinatore: 	&Nicolai Glud\\
						&Jacob Roesen\\
Scrummaster: 			&Johnny Kristensen\\
\end{tabular}
\caption{Tabel over rollefordelinger}
\label{table:roller}
\end{table}
Vi har valgt at lave roller ud fra vores udviklingsmetode, Scrum, der er beskrevet senere. Projektlederen har haft til ansvar at strukture arbejds- og Scrummøder. Projektkoordinators ansvar har været at planlægge møder og bestille lokaler. Scrummasteren er anvarlig for udviklingsplatformen, Scrum, og sørge for at metoden anvendes som tiltænkt. 