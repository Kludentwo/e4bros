\chapter*{Appendix B: Database}
\section*{Hovedvindue fro severen}
\begin{figure}[htbp]
	\centering
	\includegraphics[width=0.4\textwidth]{billeder/database/server}
	\caption{Server billed}
	\label{fig:server}
\end{figure}


\subsection*{Luk Severen}
\begin{figure}[htbp]
	\centering
	\includegraphics[width=0.4\textwidth]{billeder/database/databaseLogOff}
	\caption{Luk server}
	\label{fig:databselLogOff}
\end{figure}
Ved tryk på "Luk server" vil billedet \ref{fig:databselLogOff} fremkomme og man kan trykke "Yes" for for at lukke severen og "No" for at lade severen fortsætte med at køre.

\section*{Webside}

\section*{MySQL}
MySQL er en flertrådet SQL-database som understøtter mange samtidige brugere. MySQL er en open-source program som kan downloades på mysql.com og kan benyttes med mange forskellige operativ systemer som f.eks. WIndows, Linux og MAC OS X Lion.
Under projektet er mySQL blevet benyttet på Ubuntu(Linux version) og MAC OS X Lion. TIlgang til denne er blevet gjort med den grafiske bruger grænseflade phpMyAdmin (webbaseret) og terminalen\footnote{beskrevet under afsnittet: Kommandoer for adgang og brug af mySQL i terminal}

\subsection*{Datatyper}
MySQL understøtte følgende datatyper\footnote{Kilde mysql.com} \\
\textbf{INT} -familien:\\
\textbf{INT}: Bruges udelukkende til heltal som ikke indeholder mellemrum, linjeskift eller lignende.\\
\textbf{SmallINT}: Fungere som INT, men bruges til små tal.\\
\textbf{MediumINT}: Fungere som INT, men bruges til mellemstore tal.\\
\textbf{BigINT}: Fungere som INT, men bruges til store tal.\\

\textbf{Andre datatyper}:\\
\textbf{Varchar}: Bruges til både tal, bogstaver og enkle tegn, en linje.\\
\textbf{Char}: Bruges udelukkende til bogstaver, en linje.\\
\textbf{TinyText}: Bruges til småe resume'er, linjeskift er tilladt samt alle former for tegn og bogstaver.\\
\textbf{Text}: Bruges til mellemlange  tekster, linjeskift er tilladt og alle former for tegn og bogstaver kan benyttes.\\
\textbf{Longtext}: Bruges til lange tekster, linieskift er tilladt og alle former for tegn og bogstaver kan benyttes.\\
\textbf{Decimal}: Bruges udelukkende til decimaltal.\\
\textbf{Date}: Bruges udelukkende til at håndtere datoer. Dato formen skal være på dd-mm-år.\\

\subsection*{Kommandoer for adgang og brug af mySQL i terminal}
For at kunne benytte mySQL med password og username skal dette opsættes. Dette er blevet gjort fra terminalen ved at åbne termianlen og skrive mysql dette vil logge en på første gang. For oprettelse af bruer f.eks. root gøres følgende:\\
mysql> use mysql;\\
mysql> update user set password=PASSWOD("NEWPASSWORD") where User='root';\\
mysql> flush privileges;\\
mysql> quit

root er nu sat med password. For at logge på med root benyttes:
mysql -u root -p'password'\\
herfra er følgende kommandoer muligt:
\begin{table}[H]
\begin{tabular}{|p{5cm}|p{10cm}|} \hline
\cellcolor[gray]{0.85}Kommandoer& \cellcolor[gray]{0.85}Beskrivelse  \\ \hline
show databases; & Hviser alle databaser der er tilgængelige for den brugeren der er logget på databasen(f.eks. her ship)   \\ \hline
create database 'navn på database' & Opretter en database   \\ \hline
use 'database'; & For at vælge en bestemt database, f.eks.: use ship;    \\ \hline
show tables; & Hviser tabeller i databasen   \\ \hline
create table 'navn på tabel'('row name' char(20), 'row name' char(20));  & Opretter en tabel med et valgt navn('navn på tabel') tabellen skal vide hvor mange kolonner der skal oprettes, her en med Name og en char længde på 20   \\ \hline
DESCRIBE 'tabel navn' & Hviser den oprettede tabel   \\ \hline
select * from 'tabel' & Hviser indholdet af en tabel   \\ \hline
INSERT INTO 'tabel navn'('Indhold første row', 'indhold anden row');  & Indsætter data i den oprettede tabel under de to kolonner   \\ \hline
drop table 'tabel navn'; & Sletter tabellen\\ \hline
drop database 'navn'; & Sletter databasen\\ \hline
quit & Logger af mySQL\\ \hline
\end{tabular}
\caption{Tabel over basale kommandoer i mySQL}
\label{table:mysqlKommandoer}
\end{table}

