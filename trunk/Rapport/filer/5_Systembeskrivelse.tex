\chapter{Systembeskrivelse}
BROS er et sikkerhedssystem til skibe. Systemet aktiveres ved lastning eller losning. Her er det systemets opgave at sørge for at skibet ikke får slagsside - heraf navnet: Bias Reducing Operating System (Slagsidereducerende Operativt System).
I systemet er der indbygget en hældningssensor og to vandballasttanke - en i hver side af skibet. På baggrund af målinger fra hældningssensoren vil styringsmodulet vurdere hvorledes indholdet af tankene skal justeres af vandballasttankeenhederne således at der korrigeres for en slagside af skibet.

Hele systemet styres fra Skibsførens kontor hvor Kontrolinterfacet - en grafisk brugergrænseflade - er installeret. Her kan der aflæses skibets hældning, vandindholdet af tankene og statusmeldinger for systemet. Det er også her systemet aktiveres og deaktiveres.
Som udgangspunkt vil systemet automatisk opretholde en hældning på nul grader, men hvis man ønsker det kan man her manuelt give skibet en mindre slagside. Dette kan gøres for at imødekomme en større slagside til modsatte side påført af forestående ændringer i skibets last.

For at indsætte et ekstra sikkerhedselement vil systemet under hele processen løbende sende værdier for systemet til en ekstern database. Dermed kan en repræsentant fra terminalen følge skibets status.

\begin{figure}[H]
\centering
\includegraphics[width = .72\textwidth]{billeder/systemfigur}
\caption{Systemskitse af BROS}
\label{fig:systemskitse}
\end{figure}