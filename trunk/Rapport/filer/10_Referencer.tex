\chapter{Referencer}
\section{Artefakter}
\subsection{Kravspecifikation}
Kravspecifikationsdokumentet er udarbejdet i begyndelsen af projektet og omfatter beskrivelse af Use Cases, ikke funktionelle krav samt kvalitetsfaktorer. Den fuldstændige kravspecifikation kan se i bilag. (Kravspecifikation.pdf)

\subsection{Accepttestspecifikation}
Accepttestspecifikationsdokumentet beskriver de tests der skal laves for at undersøge om de ønskede krav er opfyldt. Den fuldstændige accepttestspecifikation kan ses i bilag (Accepttest.pdf).

\subsection{Systemarkitektur}
Systemarkitektur dokumentet beskriver systemets HW/SW opbygning og grænseflader. Den fuldstændige systemarkitektur kan ses i bilag (Systemarkitektur.pdf).

\subsection{Integrationstestspecifikation}
Integrationtestspecifikation beskriver de test der skal laves for at undersøge hvorledes de forskellige komponenter kan kommunikere. Den fuldstændige Integrationstest kan ses i bilag (Integrationstest.pdf).

\subsection{Detaljeret design}
\label{ch:DetajlDesign}
Det detaljerede design dokument beskriver hvordan HW/SW er designet og hvordan systemets komponenter fungerer. Det fuldstændige Detaljeret design dokument kan ses i bilag (Detaljeret Hardware design.pdf og Detaljeret Software design.pdf).

\subsection{Enhedstestspecifikation}
Enhedstestspecifikation beskriver de tests der skal laves for at undersøge om de forskellige stubbe af systemet fungere hensigtsmæssigt. Den fuldstændige enhedstestspecifikation kan ses i bilag (Enhedstest.pdf).

\section{Hjemmesider}
http://www.docs.google.com\\
http://office.microsoft.com/en-us/visio/\\
http://www.maplesoft.com/\\
http://www.ni.com/multisim/\\
http://kurser.iha.dk/eit/eit-lab/lagerliste.htm\\
https://www.retsinformation.dk/Forms/R0710.aspx?id=26449\\
http://www.apache.org/\\
http://www.mysql.com/\\

\section{Liste over bilag på CD}
Komponentliste.pdf\\
SCRUM.xls\\
Logbog.pdf\\
\subsection{Kode}
\subsubsection{KI}
hest
\subsubsection{SM}
Haeldningsregulering.c.pdf
Haeldningsregulering.h.pdf
Haeldningssensorblok.c.pdf
Haeldningssensorblok.h.pdf
Init.c.pdf
Init.h.pdf
Kommunikation.c.pdf
Kommunikation.h.pdf
komnavn.h.pdf
main.c.pdf
protokolenum.h.pdf
smflags.h.pdf
VBTE-Driver.c.pdf
VBTE-Driver.h.pdf
sm.rar (PSoC Creator 2.1 Projekt i et rar arkiv)
\subsubsection{VBTE}
honning
\subsubsection{Databasen}
\subsection{Dokumentation}
Accepttest.pdf\\
Arkitektur.pdf\\
Detaljeret\_hardware\_design.pdf\\
Detaljeret\_software\_design.pdf\\
Enhedstest.pdf\\
Integrationstest.pdf\\
Kravspecifikation.pdf\\

\subsection{Datablade}
\begin{table}[H]
\begin{tabular}{|l|l|}
\hline
\textbf{Datablad:}& \textbf{Type:} \\\hline
CY8C55 & PSoC5 \\\hline
KXSC7-2050& Accelerometer\\\hline
ST3232cn & Level konverter \\\hline
BD139 & Transistor \\\hline
BC547 & Transistor \\\hline
400SRT 160 & Ultralydstransmitter og receiver \\\hline
Ventil & Ventil til ventilspolen \\\hline
Ventilspole & Ventilspole der driver ventilen \\\hline
LM317 & Spændingsregulator \\\hline
kbl04 & Diodebro \\\hline

\end{tabular}
\end{table}


\subsection{Billeder}
hvis vi har billeder af vores produkt!\\

\subsection{Jura}
Gruppen har inden start lavet en grundig undersøgelse af de juridiske problemstillinger som opstår i forbindelse med et projekt som BROS. De vigtigste dele er blevet vedlagt i bilag Jura.