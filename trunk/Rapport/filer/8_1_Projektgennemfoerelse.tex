\section{Projektgennemførelse}
Projektet er udført af en gruppe på fem personer. Gruppen er ny og sammensat af personer fra fire forskellige projektgrupper. Dette har medbragt både gode som dårlige ting. Det er godt at opdage nye måder at anskue et projektforløb og nye vinkler på udfordringer i processen. Det har dog desværre kostet en del energi at skulle tilpasse sig hinanden. En proces hvor alle har måtte gå på kompromis på et eller flere punkter.

Længden af faserne har vist sig at være skæve da processen ikke har været lige så gnidningsfri som først håbet. Som følge heraf er tidsplanen blevet revirderet flere gange og opgaver fra ét SCRUM har måttet videreføres i det næste.

Projektets overordnede tidsplan for faserne og de eksterne milestones ligger som bilag.

\subsection{Rollefordelinger}
\begin{table}[H]
\centering
\begin{tabular}{|l|l|} \hline
Projektleder: &Jacob Roesen\\\hline
Projektkoordinator: &Nicolai Glud\\
	&Jacob Roesen\\\hline
Scrummaster: &Johnny Kristensen\\\hline
\end{tabular}
\caption{Tabel over rollefordelinger}
\label{table:roller}
\end{table}
Vi har valgt at lave roller ud fra vores udviklingsmetode, SCRUM, der er beskrevet senere. Projektlederen har haft som ansvar at strukture arbejds- og scrummøder. Projektkoordinators ansvar har ligget i at planlægge møder og bestille lokaler. Scrummasteren er anvarlig for udviklingsplatformen, SCRUM, og sørge for at metoden anvendes mest optimalt. 