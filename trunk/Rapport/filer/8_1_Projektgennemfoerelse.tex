\section{Projektgennemførelse}
Projektet er udført af en gruppe på fem personer. Gruppen er ny og sammensat af personer fra fire forskellige projektgrupper. Grundet mange erfaringer fra de forskellige grupper har gruppen valgt at arbejde med projektet ud fra scrum princippet. Gruppen har lagt roller fast \ref{table:roller} og arbejdsmetoder. Projektet er blevet brudt ned i mindre dele og gruppens enkle medlemmer har fået ansvar for hver deres del og sparet med den i gruppen som blokken kommunikere med.

Længden af faserne har vist sig at være skæve da processen ikke har været lige så gnidningsfri som først håbet. Som følge heraf er tidsplanen blevet revirderet flere gange og opgaver fra ét Scrum har måttet videreføres i det næste.

Projektets overordnede tidsplan for faserne og de eksterne milestones ligger som bilag.

\subsection{Rollefordelinger}
\begin{table}[H]
\centering
\begin{tabular}{|l|l|} \hline
Projektleder: &Jacob Roesen\\\hline
Projektkoordinator: &Nicolai Glud\\
	&Jacob Roesen\\\hline
Scrummaster: &Johnny Kristensen\\\hline
\end{tabular}
\caption{Tabel over rollefordelinger}
\label{table:roller}
\end{table}
Vi har valgt at lave roller ud fra vores udviklingsmetode, Scrum, der er beskrevet senere. Projektlederen har haft som ansvar at strukture arbejds- og scrummøder. Projektkoordinators ansvar har ligget i at planlægge møder og bestille lokaler. Scrummasteren er anvarlig for udviklingsplatformen, Scrum, og sørge for at metoden anvendes mest optimalt. 