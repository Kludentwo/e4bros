\chapter{Konklusion}
\label{ch:konklusion}
%\subsection{Bulletpoints til Konklusion}
Gruppen ser i slutningen af semesteret tilbage på et godt projektforløb. \\
Forløbet har givet en væsentlig forbedring af forståelsen for udviklingsprocessen. Både forståelsen af faserne og deres sampspil er blevet udbygget. På den front må projektet siges at have været meget succesfuldt.\\
Gruppen har i accepttesten bevist, at den har formået at implementere langt størstedelen af den ønskede funktionalitet. At hele accepttesten ikke har kunnet godkendes skyldes problemer med afstandsmålingen, der ellers opererede godt under kontrollerede forhold. Derfor må implementeringen siges at være delvist succesfuld.\\
Gruppen havde i starten en udfordring i at være sammensat af fem personer fra fire forskellige grupper. Denne udfordring blev vendt til en fordel, da gruppen begyndte at udnytte, at den anskuede udfordringer med fire forskellige erfaringssæt. Dette har i høj grad bidraget til læringsprocessen, og samarbejdet må dermed konkluderes at have været succesfyldt.\\

\chapter{Forbedringer til systemet}
\label{ch:Forbedringer}
I dette afsnit samles op på det endelige system, og der diskuteres forbedringer til systemet. Der diskuteres ikke udvidelser, men reelle forbedringer til det dokumenterede system. Afsnittet er delt op i fire punkter: \textit{Sikkerhed}, \textit{Optimering af overførsel af data}, \textit{Effektivitet} og \textit{Prototype}. 

\begin{large}\begin{itemize}
\item Sikkerhed
\end{itemize}\end{large}
Før systemet skulle implementeres mere er der meget sikkerhed der skal tages til overvejelse. Visse scenarier vil få katastrofale konsekvenser hvis systemet kom ud på et stort skib. Forsvandt forbindelsen mellem SM og VBTE mens en VBTE er ved at fylde i en tank vil dennes tilstand ikke blive ændret og VBTE'en vil blive ved med at fylde vand ind til der ikke kunne være mere og skibet ville kæntre. Forhåbentligt ville SM og den anden VBTE, hvis der stadig var forbindelse der, kunne oprette holde vatter for skibet ved at fylde denne tank også. Men forsvandt SM fuldstændigt kunne det gå helt galt.\\
Her kunne indføres timere på hver VBTE til at lukke for ventilerne hvis der ikke blev opdateret tilstand i en hvis tidsperiode. En slags watchdog.
\begin{large}\begin{itemize}
\item Optimering af overførsel af data
\end{itemize}\end{large}
Sådan som systemet lige nu håndterer data bliver der udskrevet til brugeren, at forbindelsen er mistet til Styringsmodulet, hvis der bliver svaret med forkert data til Kontrolinterfacet. Dette kan optimeres ved at forsøge flere gange, hvis det rigtige svar ikke kommer tilbage.\\
Derudover er der ikke nogen validering af data mellem Styringsmodulet og Vandballasttankenhederne. Den eneste sikring indført her er at ventilerne bliver lukket hvis der bliver modtaget data uden for protokollen. Man kunne indføre at VBTE svarer tilbage med hvad den har sat sig til og herefter sender niveauet tilbage.
\begin{large}\begin{itemize}
\item Effektivitet
\end{itemize}\end{large}
Systemet tømmer/fylder kun en tank af gangen som implementeringen er sket. Dette kunne optimeres på flere måder. Den ene er mere intelligent software på Styringsmodulet. Den anden, og mere interessante, kunne være helt at undlade at implementere Styringsmodulet og have en hældningsmåler på hver Vandballasttankenhed. Vandballasttankenheden vil derfor reagere selvstændigt på hældningen resulterende i at der kan blive både tømt og fyldt på samme tid i hver sin side. Problemet med dette bliver så kontakt til Kontrolinterfacet og at der skal laves væsentlige ændringer i hele struktureringen af systemet.\\
Til effektivitet kan også nævnes at manuel hældningsregulering også manuelt skal sættes tilbage igen. Dette kunne håndteres ved hjælp af Styringsmodulet. Hvis det var nødvendigt med manuel hældning ville man også kunne antage at systemet ville ramme tilbage i vatter i takt med den store lastning. Dermed burde Styringsmodulet automatisk aktivere automatisk hældningsregulering når hældningen igen er vatter. Desuden kunne et check på om Databasen har modtaget data fra Kontrolinterfacet sikre at terminalrepræsentanten har mulighed for at opdage en mulig fejl.
\begin{large}\begin{itemize}
\item Prototype
\end{itemize}\end{large}
Der er i projektet anvendt ventiler til at styre in/out-flow af vand til tankene. Dette er anvendt, da det var dem vi kunne få af Danfoss. Dette vil ikke fungere på et skib og det gør også reguleringen af vores prototyper langsommere. Det vil derfor være meget bedre at anvende pumper til systemet.