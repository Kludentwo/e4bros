\section{Design og Implementering}
\label{ch:DesignImplementering}
%%%%%%%%%%%%%%%%%%%%%%%%%%%%
%%%%%%%%%% VBTE %%%%%%%%%%%%
%%%%%%%%%%%%%%%%%%%%%%%%%%%%
\subsection{VBTE}
I dette afsnit beskrives design og implementering af VBTE for både software og hardware. 
\subsubsection{Hardware}
Til VBTE'en er der blevet designet hardware til at styre de to ventiler samt den keramiske ultralydstransmitter og receiver. Hardware er blevet udfærdiget i to dele. Den ene er programeret hardware på PSoC'en, den anden er hardware uden for PSoC'en. Hardware designprocessen til VBTE'en gik igennem 3 faser:
\begin{enumerate}
\item Overordnet design
\item Nedbrydning af blokke
\item Opbygning af design
\end{enumerate}
Gennem disse faser er designet blevet udfærdiget. Fremgangsmåden er anvendt for at overskueligtgøre systemet og lette arbejdet ved at dele systemet op i små dele. På \textit{figur \ref{fig:HWVBTE}} ses det overordnede design af VBTE'en. Der vil i rapporten tages udgangspunkt i ventilkredsen samt receiverdriveren på PSoC'en.
\begin{figure}[H]
\centering
\includegraphics[width = 0.9\textwidth]{billeder/HWVBTE}
\caption{Illustrering af overordnet design af hardware på VBTE.}
\label{fig:HWVBTE}
\end{figure}
\subsubsection{Hardware på PSoC'en}
Hardware opbygget på PSoC'en kunne uden problemer være blevet designet uden for PSoC'en, men PSoC miljøet gør det meget nemt at arbejde med mange forskellige elementer. Der vil i rapporten ligges vægt på transmitterkredsen samt receiverkredsen. På figur \ref{fig:HWVBTE} ses hardware designet på PSoC'en.
\begin{figure}[H]
\centering
\includegraphics[width=0.7\textwidth]{billeder/PSOC_Hardware}
\caption{Hardware på PSoC'en.}
\label{fig:PSOCHW}
\end{figure}
\subsubsection{Transmitterkreds}
Transmitterkredsen står for at sende burst samt at time hvert burst.  Dette opnåes med to clocks, en AND gate, en output pin samt et kontrol register. Transmitterclocken er indstillet til 40kHz da ultralydstransmitteren, ifølge databladet, opererede ved $\SI{40}{kHz}\pm\SI{1}{kHz}$. Clocken er blevet målt på oscilloskop til $\SI{40.3}{kHz}$. Burst kontrolregisteret er anvendt til at AND'e clocken ud på trans\_kontrol pinden.\\
Clock\_dist interruptrutinen sørger for at tælle en variabel op der anvendes i main til at kalde burst funktionen. Dette gøres for at lave et nonblocking delay så andet kan køres på PSoC'en mens en burst er sendt afsted og der afventes detektion.
\subsubsection{Receiverkreds}
Receiverkredsen modtager signalet fra ultralydsreceiveren og omsætter det til en detektion. Dette sker efter følgende opskrift:\\
Løfte signalet $\rightarrow$ Forstærkning $\rightarrow$ Mixning.\\
Signalet bliver løftet på til $\frac{Vdda}{2}$ fordi PSoC'en ikke kan arbejde med værdier under Vssa (GND). Dette gøres ved hjælp af en kondensator, en modstand og en spændingsfølger med $\frac{Vdda}{2}$ på det positive ben.\\
Efter signalet er løftet bliver det forstærket op af en PGA. Efter undersøgelser, hvor der blev sendt burst, er der valgt en forstærkning på 16, da der ca. modtages et signal på $\SI{200}{mV} p-p$.\\
Herefter bliver signalet mixet med $\SI{40}{kHz}$ og filtreret. Efter mixeren giver det, stort set, en DC og summen af de to signaler. Filteret er inbygget i Delta-Sigma ADC'en og har en knækfrekvens ved $\frac{Sample frekvens}{3}$. For at være sikker på at det meste af signalet er kommet over regnes der en opladningstid på filteret til $\SI{1/4}{ms}$. Efter undersøgelser af signalet ind på ADC'en blev en spænding på $\SI{0.3}{V}$.

\subsubsection{Hardware uden for PSoC'en}
Uden for PSoC'en er der lavet 2 hardwareblokke. Disse har til ansvar at omsætte et kontrolsignal til en større spænding over de respektive enheder.
\begin{figure}[H]
	\centering
	\begin{minipage}[b]{0.48\textwidth}\centering
	\includegraphics[width=0.80\textwidth]{billeder/Ventilblok}
	\end{minipage}
	\begin{minipage}[b]{0.48\textwidth}\centering
	\includegraphics[width=0.80\textwidth]{billeder/Transmitterblok}
	\end{minipage}
	\begin{minipage}[t]{0.48\textwidth}
	\caption{Hardwareblok for ventil}
	\label{fig:SMHW1}
	\end{minipage}
	\begin{minipage}[t]{0.48\textwidth}
	\caption{Hardwareblok for transmitteren}
	\label{fig:SMPSOC1}
	\end{minipage}
\end{figure}
\subsubsection{Ventilkreds}
Ventilkredsen får to kontrolsignaler fra PSoC'en og disse skal styre de to ventiler. Ventilerne monteret på kredsen er fra Danfoss og er af modellerne EV210A-1.2 og EV210A-4.5. Disse ventiler drives ved 12V 0.4A. Der er valgt en BD139 transistor til at forstærke signalet op. Denne har en forstærkning mellem 40 og 160 (aflæst fra databladet\footnote{Se bilag/BD139.pdf}). Dette kan kun lige drives med en BD139 transistor hvilken har en forstærkning på 40-160. Der er derfor lavet en darlington kobling for at mindske belastningen af PSoC'en og for at sikre at der bliver lukket nok op for transistoren. Transistorne er valgt ud fra pris og tilgængelighed. Der var først valgt en MOSFET IRLZ44n men denne var både for dyr og kunne klare en unødvendig stor effekt. 
\subsubsection{Software}


%%%%%%%%%%%%%%%%%%%%%%%%%%%%
%%%%%%%%%%% SM %%%%%%%%%%%%%
%%%%%%%%%%%%%%%%%%%%%%%%%%%%
\subsection{SM}
I dette afsnit beskrives design og implementering af SM modulet. SM modulet består af både software og hardware.
\subsubsection{Hardware}
SM modulets hardware består af en konverteringskreds og en PSoC. På PSoC'en er monteret et Kionix KXSC7-2050 accelerometer. Konverteringskredsen anvendes til at sende UART signaler fra PSoC til en KI modulet. Accelerometerets x-akse anvendes til hældningsmålinger for hældningssensorblokken. Designfasen til SM er delt op i 3 faser:
\begin{enumerate}
\item Overordnet design
\item Nedbrydning af blokke
\item Opbygning af design
\end{enumerate}
Denne fremgangsmåde gør det muligt for en udefrakommende at følge med i processen og at kunne implementere modulet så det overholder de krav der er stillet. Ligeledes gør fremgangsmåden det lettere at overskue flere løsninger til hvert problem. på \textit{Figur~\ref{fig:SMHW1}} ses det Overordnede design og på \textit{Figur~\ref{fig:SMPSOC1}} ses PSoC blokken i SM. Der bliver efterfølgende taget udgangspunkt i Hældningssensoren. \\
\begin{figure}[H]
	\centering
	\begin{minipage}[b]{0.48\textwidth}\centering
	\includegraphics[width=0.80\textwidth]{billeder/SMHardware}
	\end{minipage}
	\begin{minipage}[b]{0.48\textwidth}\centering
	\includegraphics[width=0.80\textwidth]{billeder/SMPSoCblock}
	\end{minipage}
	\begin{minipage}[t]{0.48\textwidth}
	\caption{Hardware blok for SM}
	\label{fig:SMHW1}
	\end{minipage}
	\begin{minipage}[t]{0.48\textwidth}
	\caption{PSoC blok for sm}
	\label{fig:SMPSOC1}
	\end{minipage}
\end{figure}
Hældningssensoren består af 2 komponenter: det førnævnte accelerometer samt en DelSig ADC internt i PSoC'en. Valget af accelerometer kommer af at have lavet en række prototyper der ikke mødte vores krav, hvilket accelerometeret i PSoC'en gjorde. Valget af ADC faldt på en DelSig da, den er meget støj immun grundet det indbyggede lavpas filter og har en høj opløsning. Valgte komponenter er illustret på \textit{Figur~\ref{fig:levelsensor}}. ADC konverterer det analoge signal fra, en pin forbundet til, accelerometer til en digital værdi der så senere bliver anvendt i softwaren. For ADC og accelerometerets opsætning se da afsnit \ref{ch:DetajlDesign}~\textit{Detaljeret hardware design} i Bilag.
\begin{figure}[htbp]
	\centering
	\includegraphics[width=0.50\textwidth]{billeder/levelsensor}
	\caption{Hældningssensorens implementering}
	\label{fig:levelsensor}
\end{figure}
\subsubsection{Software}