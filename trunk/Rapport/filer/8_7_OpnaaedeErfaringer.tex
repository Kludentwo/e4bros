\section{Opnåede erfaringer}
\label{ch:OpXP}
%%%%%%%%%%%%%%
%\subsection{bulletpoints til OpXP}
%\item Nyt samarbejde og nye gruppemedlemmer (Hvor godt vi har sammensat en ny gruppe)
%\item Hvor mange iterationer vi har været igennem (Forbedringer af start design). Igen vigtigheden af at man altid kan gå tilbage og at dokumentation ikke er støbt i beton.
%\item Opnåede erfaring omkring faserne (Hvor vigtige de er og hvordan vi efterfølgende har haft det nemmere på faserne)
%\item Samarbejde med kunden. Inddrage kunden(Carl) i processen.
%\item Forståelse for test (Test er ikke til for bare at vise systemet virke men for at afsløre væsentlige mangler, hvilket har forbedret projektet (og produktet)).
%\item Uddeligering af opgaver (Noget johnny sagde med at det var vigtigt der var en ansvarsperson bag hver modul).
%\item Udvikling (Sammenkogt erfaring af Johnny og Nicolai)
%\end{itemize}
%%%%%%%%%%%%%%%

\subsection{Generelt}
Dette projekt er udarbejdet af en gruppe bestående af medlemmer fra fire grupper. Hver gruppe havde deres erfaringer med projektforløb og det at kombinerer erfaring for at opnå et mere succesfuldt forløb er en erfaring denne gruppe har gjort. Endvidere har gruppen formået et godt samarbejde med indgangsvinkler baseret på tidligere forløb.

\subsection{Agile udviklingsmetoder}
Gruppen har gennem dette projekt forløb fået en bedre forståelse for iterationer og hvordan dokumentation holdes opdateret efter behov. Konkret agerede gruppens vejleder som kunde, hvilket muliggjorde interaktion med "kunden". Gruppen har erfaret at en kundes behov og forventninger kan skifte i forbindelse med et projekt. Endvidere har gruppen erfaret vigtigheden i omstilling og muligheden for arbejde iterativt.\\
Gruppen har arbejdet med faserne i projekt og har opnået en bedre forståelse for vigtigheden af faser og iterationer. Gennem en iteration er kravspecifikation og systemarkitekturen blevet forbedret, hvilket efterfølgende har gjort implementering og tests lettere at håndtere og gennemføre.\\
I udviklingsforløbet er systemets moduler blevet delt op i ansvarsområder. Gruppen har erfaret at dette er en force, da det sikrer et tilhørelsesforhold og dermed også et ansvar for hver enkelt moduls opgaver blev udført.

\subsection{Udvikling af nye Komponenter}
Gennem udvikling af projektet har gruppen erfaret, at udviklingen at nye moduler kan være så tidskrævende at det ofte kan svare sig at købe færdiglavede komponenter og inkorporerer dem i systemet for at sikrer en overordnet funktion. Udviklingstiden for et modul vurderes ud fra teknologiundersøgelsen og da gruppen ikke har prøvet at lave et komponent helt fra bunden er dette en ny erfaring.

\subsection{Test}
Fra tidligere projekter har gruppen tænkt tests som noget der bare skulle laves. Dette har medført et ambivalent forhold til test. I dette projekt har gruppen erfaret at test er med til at fremhæve mulige forbedringer af den implementering, der er udført. Testresultater medtages i den iterative designprocedure, med henblik på at forbedre det testede modul. Denne erfaring har ført til mere tilfredstillende prototyper igennem projektforløbet.