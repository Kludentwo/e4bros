\section{Systemarkitektur}
\subsection{Udkast fra Lund}
I dette afsnit vil systemarkitekturen for projektet blive beskrevet. Systemarkitekturen tager udgangspunkt i dokumentationsdokumentet "Systemarkitektur" og for nærmere uddybning henvises der til det dokument. 
Systemarkitekturen er udarbejdet på grundlag af kravspecifikationen og systemet som beskrevet i projektbeskrivelsen.

Systemarkitekturen starter med funktionaliteten beskrevet i Use Casene og udvider så beskrivelsen til at omfatte de elementer af systemet som brugeren ikke interagerer med. Dette afsnit vil give et eksempel på hvordan en Use Case er blevet behandlet i systemarkitekturen. Use Casen der vil blive brugt som eksempel vil være Use Case 2: Skibsfører slår automatisk hældningsregulering til.

\fxnote{indsæt domænemodel}
\fxnote{indsæt applikationsmodel}
\fxnote{indsæt sekvensdiagram}\\
I domænemodelen beskrives relationen mellem systemets elementer og aktørerne. Disse enheder omdannes i applikationsmodelen til konceptuelle klasser af en given type. Det ses at i Use Case 2 agerer KI som både en kontrolklasse og en grænsefladeklasse. Den agerer som en kontrolklasse fordi den er initiativtager og beslutningstager til den efterfølgende udvikling i systemet. Både KI, SM og VBTE agerer som en grænsefladeklasse fordi alle tre klasser er i berøring med en aktør.\\
I sekvensdiagrammet beskrives kommunikationen og timingen imellem klasserne. Det ses at brugeren igangsætter processen, KI videresender kommandoen, SM bekræftiger modtagelsen af kommandoen hvorefter den begynder at regulere på vandniveauet i tankene ved hjælp af VBTE'erne. Reguleringen sker på grundlag af de målinger den modtager fra hældningssensoreren. Vandniveaet vil blive ved med at blive reguleret med et mål om at nå og opretholde en hældning på nul grader.

\fxnote{indsæt statemachine}\\
Når alle Use Cases er blevet bearbejdet som Use Case 2 er blevet ovenfor har gruppen dannet state machines for de konceptuelle klasser i systemet. Her visualiseres hvilke stadier klasserne har brug for at gennemgår i løbet af klassens levetid for at opfylde kravene sat i kravspecifikationen.

\subsection{Tidl. text}
Dette afsnit beskriver systemarkitekturen for for projektet ”BROS" som formuleret i projektbeskrivelsen og specificeret i kravspecifikationen.
Afsnittet indeholder beskrivelse af systemkomponenter, systemarkitektur, SW-komponenter, HW-komponenter og interfaces, i den givende rækkefølge.

\subsection{Systemkomponenter}
Ud fra kravspecifikationen er der udvalgt disse beslutninger om komponenter til systemet og deres placering. Der refereres derfor til kravspecifikationens, ikke-funktionelle krav og krav generelt.

Brugeren integrere med systemetet igennem KI. KI er styringsmodulet for hele systemet. På KI har brugeren mulighed for at til- og frakoble systemet, justere den ønskede hældning på skibet. KI giver mulighed for at brugeren kan aflæse handlinger foretaget i systemet samtidig med at denne modtager advarsler i tilfælde at hældningen bliver for stor eller vandbalast tanke bliver overfyldte.\\
KI kommunikere til SM modulet igennem en uart. for at denne kommunikation kan foregå er der lavet en protocol for denne kommunikation. Kommunikationsformen er ved I$_2$C.\\
SM står for at måle skibets hældning og sende denne til KI. KI kan så informere dette til brugeren. Når SM har målt hældningen på skibet giver denne besked til VBTE1 og VBTE2 om at åbne og lukke for ventilerne til tankene. VBTE1 og VBTE2 styres fra en PSoC. VBTE1 og VBTE2 er placeret på være deres tank. For at kunne kontrollere hvor meget vand der er i tankende dette gøres ved hjælp af to ultra lydssensore som hele tiden måler og vidergiver denne information til SM som så sender dette videre til KI der kan advare om vandstanden i tankene i tilfælde af at systemet af sat på manuel styring.
KI sender data om skibet til databasen som lagre disse data i en mySQl database som så kan tilgås via et web interface.